\subsection{Účel a~struktura bezpečnostní zprávy}

Tato bezpečnostní zpráva je v~pořadí šestým vydáním bezpečností zprávy školního reaktoru VR-1 a~nahrazuje předchozí vydání provozní 
bezpečnostní zprávy z~roku 2007 \cite{bezp_zprava_2007}, resp. její revizi z~roku 2014 \cite{bezp_zprava_2014}. Důvodem pro zpracování 
nového vydání bezpečnostní zprávy reaktoru VR-1 byla především příprava žádosti o~povolení provozu reaktoru s~cílem zohlednit:

\begin{itemize}
    \item	významné změny v~české „atomové“ legislativě a~s~tím související změny v~provozní a~bezpečnostní dokumentaci reaktoru, 
    \item	výstupy aktualizovaných bezpečnostních analýz,
    \item	změny na některých technologických zařízeních reaktoru.
\end{itemize}

Přestože se bezpečnostní zpráva neřadí mezi schvalovanou dokumentaci, je součástí dokumentace předkládané SÚJB v~rámci správního řízení 
k~vydání povolení k~provozu reaktoru VR-1. Účelem bezpečnostní zprávy je poskytnout aktuální a~komplexní informace o~reaktoru a~jeho 
lokalitě, které umožní jednoznačně posoudit a~ověřit, že provoz tohoto zařízení je bezpečný a~nepředstavuje nepřípustné riziko pro 
zdraví a~bezpečnost jeho obsluhy, obyvatelstva a~nemá negativní dopad na životní prostředí. \par
Struktura a~členění bezpečnostní zprávy vychází z~doporučení Mezinárodní agentury pro atomovou energii (dále jen IAEA), které jsou 
zpracovány ve formě bezpečnostního návodu č. SSG-20 \cite{ssg-20}. Návod se věnuje provádění bezpečnostních analýz a~tvorbě 
bezpečnostní zprávy pro výzkumné reaktory, přičemž je zde detailně popsán obsah bezpečnostní zprávy. V~souladu s~tímto návodem je 
bezpečnostní zpráva reaktoru VR-1 rozdělena do 20 kapitol, jejichž obsah odpovídá nejen doporučením IAEA, ale plně respektuje
i~požadavky SÚJB, které jsou stanoveny vyhláškou o~požadavcích na projekt jaderného zařízení (viz příloha 3 této vyhlášky) 
\cite{vyhlaska_XXX_2017}. \par

Podklady využité při zpracování této bezpečnostní zprávy lze rozdělit do čtyř základních skupin:
\begin{enumerate}
    \item Interní dokumentace
    \item	Technická a~výkresová dokumentace
    \item	Legislativa
    \item	IAEA dokumenty
\end{enumerate}

Dle výše uvedeného členění jsou dokumenty použité při tvorbě bezpečnostní zprávy shrnuty v~kapitole s~názvem literatura. \par
První kategorie zahrnuje kompletní dokumentaci, která souvisí s~provozem reaktoru. Jedná se o~jednotlivé verze bezpečnostních zpráv, 
limity a~podmínky, program provozních kontrol, provozní předpisy, řídicí postupy, návody, směrnice, řády a~další. Využívána byla 
jak aktuální dokumentace, tak i~dokumentace původní. Do druhé kategorie spadá technická a~výkresová dokumentace, z~níž bylo čerpáno 
při tvorbě bezpečnostních zpráv. Ve třetí kategorii jsou zařazeny dokumenty české legislativy, tj. zákony a~vyhlášky. Částečně byly 
využity také návody SÚJB a~výnos Československé komise pro atomovou energie (ČSKAE, předchůdce SÚJB) č. 9 z~roku 1985. Čtvrtá kategorie
 obsahuje návody a~doporučení IAEA především z~oblasti provozu a~využívání výzkumných reaktorů.

\subsection{Držitel povolení}

Držitelem povolení k~provozu školního reaktoru VR-1 je České vysoké učení technické v~Praze (dále jen ČVUT). ČVUT je veřejnou vysokou 
školou univerzitního typu, v~souladu se zákonem o~vysokých školách č. 111/1998 Sb. \cite{zakon_111_1998} se jedná o~právnickou osobu.
 Statutárním zástupcem ČVUT je rektor, který je zároveň vrcholným reprezentantem univerzity ve vztahu k~jiným vysokým školám v~České 
 republice i~zahraničí, k~veřejným institucím a~státním orgánům, zejména MŠMT, podnikatelské sféře i~občanům \cite{statut_CVUT_VI}. \par
Provoz reaktoru zajišťuje Fakulta jaderná a~fyzikálně inženýrská (dále jen FJFI), konkrétně katedra jaderných reaktorů. Pracoviště 
reaktoru se nachází v~těžkých laboratořích areálu Trója, Matematicko-fyzikální fakulty Univerzity Karlovy (MFF UK) v~ulici 
V~Holešovičkách 2, v~Praze 8.

Základní identifikační údaje držitele povolení jsou shrnuty v~tab.~\ref{tab:identifikacni_udaje}.

\begin{table}[hb!] \catcode`\-=12 % aby fungovaly cline
\renewcommand{\arraystretch}{1.3}%
\caption{Identifikační údaje žadatele. \label{tab:identifikacni_udaje}}
\centering
\begin{tabular}{|m{6cm}|m{10cm}|}
\hline
Název: & České vysoké učení technické v~Praze \\
\hline
Právní forma: & veřejná vysoká škola – právnická osoba \\
\hline
Sídlo: & Zikova 1903/4, 166 36 Praha 6 \\
\hline
IČ: & 68407700 \\
\hline
Adresa datové schránky: & p83j9ee \\
\hline
Provozované jaderné zařízení: & výzkumné jaderné zařízení – školní reaktor VR-1 \\
\hline
Evidenční číslo SÚJB: & 115479 \\
\hline
Lokalita jaderného zařízení: & budova těžkých laboratoří v~areálu MFF UK, V~Holešovičkách 2, 180 00 Praha 8 \\
\hline
\end{tabular}
\end{table}

\subsection{Území umístění reaktoru}
Územím umístění reaktoru je myšlena oblast popsaná v~kap.~\ref{sec:areal} a~vyobrazena na obr.~\ref{fig:areal}. V~přílohách může být tato oblast popsána také termínem \uv{lokalita}.

\subsection{Základní informace a~obecný popis reaktoru}	
Školní jaderný reaktor VR-1 je výzkumným jaderným zařízením, které je využíváno především k~pedagogickým a~vědecko-výzkumným účelům.
 Reaktor spadá do kategorie lehkovodních reaktorů bazénového typu, pracujících s~obohaceným uranem. Vzhledem k~jeho nízkému výkonu se 
 řadí mezi jaderná zařízení tzv. nulového výkonu. Reaktor je umístěn v~hale, která je součástí budovy těžkých laboratoří v~areálu 
 Trója MFF UK.

\subsubsection{Historie reaktoru}	
Projekt školního reaktoru VR-1 vypracoval Chemoprojekt Praha na základě technických podkladů, které připravila ŠKODA JS Plzeň a~Fakulta 
jaderná a~fyzikálně inženýrská (FJFI). Vlastní výstavba reaktoru byla zahájena v~roce 1985. Stavební práce zajišťovaly Pozemní stavby 
Praha, hlavní technologickou část reaktoru dodala ŠKODA JS Plzeň. FJFI se na výstavbě reaktoru podílela koordinací prací, výpočty 
neutronově-fyzikálních charakteristik reaktoru, bezpečnostními analýzami, koncepčním návrhem ovládacího zařízení reaktoru, včetně 
sestavení a~oživení jeho řídicího systému. Původní palivo pro reaktor, typově označené IRT-2M (obohacení 36 \% $^{235}$U), bylo 
dodáno z~bývalého SSSR. Palivo bylo na pracoviště reaktoru přivezeno v~letech 1988 a~1989. Ke konci roku 1988 byla dokončena výroba 
reaktorových nádob ve ŠKODA JS Plzeň a~po kontrolní montáži hlavní sestavy byl začátkem roku 1989 zahájen jejich postupný transport 
do Prahy, kde byly následně uloženy do připravených pouzder ve stínění reaktoru. Veškeré stavební a~montážní práce byly dokončeny
v~červnu 1989. Do září 1990 probíhalo neaktivní vyzkoušení, na které navázalo fyzikální spouštění. Poprvé dosáhl reaktor kritického 
stavu dne 3. prosince 1990 v~16:25 hod SEČ. Následoval zkušební provoz, během něhož byly experimentálně ověřeny všechny hlavní 
parametry reaktoru a~připravené pedagogické úlohy. \par
Reaktor VR-1 je v~trvalém provozu od ledna 1992. Časový přehled výstavby a~jednotlivých etap uvádění reaktoru do provozu je uveden 
v~tab.~\ref{tab:milniky_reaktor}. 

\begin{table}[hb!] \catcode`\-=12 % aby fungovaly cline
\renewcommand{\arraystretch}{1.3}%
\caption{Jednotlivé milníky při výstavbě a~uvádění reaktoru VR-1 do provozu. \label{tab:milniky_reaktor}}
\begin{tabular}{|m{10cm}|m{6cm}|}
\hline
 Vypracovaná zadávací bezpečnostní zpráva: &  říjen 1982\\
\hline
 Vypracovaný jednostupňový projekt: &  prosinec 1982\\
\hline
 Vypracovaná předběžná bezpečnostní zpráva: &  březen 1983\\
\hline
 Vydání stavebního povolení: &  leden 1984\\
\hline
 Schválený jednostupňový projekt: &  leden 1985\\
\hline
 Stavební a~montážní práce: &  duben 1985 – červen 1989\\
\hline
 Neaktivní vyzkoušení: &  červen 1989 – září 1990\\
\hline
 Fyzikální spouštění: & listopad 1990 – prosinec 1990 \\
\hline
Dosažení prvního kritického stavu: & 3. prosince 1990\\
\hline
Zkušební provoz: & leden 1991 – prosinec 1991\\
\hline
Trvalý provoz: & od ledna 1992\\
\hline
\end{tabular}
\end{table}


\subsubsection{Popis reaktoru}	
Reaktor VR-1 je tvořen tělesem ve tvaru osmistěnu, které je vyrobeno ze stínicího betonu. V~tělese reaktoru jsou instalovány dva 
bazény – nádoby, značené H01 a~H02. Obě jsou prakticky shodné, různá je však jejich funkce a~tím i~vybavení vnitřními částmi. První 
nádoba – H01 je reaktorovou nádobou, jejíž součástí je aktivní zóna. Druhá nádoba – H02 slouží jako manipulační nádoba. Toto uspořádání 
bylo zvoleno především z~důvodů radiační ochrany a~usnadnění některých manipulací. Manipulační nádoba umožňuje plnit řadu pomocných 
funkcí, mimo jiné je vybavena chránilištěm pro mokré skladování palivových článků. V~případě potřeby lze obě nádoby pomocí hradítka 
vodotěsně oddělit. \par
Vnitřní části reaktoru se skládají z~několika funkčních skupin, které vesměs navazují na aktivní zónu reaktoru. Patří mezi ně nosný 
systém aktivní zóny, rošty, nosný systém regulace, měřicí kanály, provozní a~měřicí potrubí a~chrániliště palivových článků
v~nádobě H02. V~reaktorové nádobě je i~plošina, která umožňuje manipulace v~aktivní zóně a~jejím okolí při snížené hladině vody.
Reaktorové nádoby jsou zaplněny lehkou vodou, která slouží nejen jako moderátor a~reflektor, ale i~jako biologické stínění a~chladivo. 
Díky malému výkonu, je pro odvod tepla uvolněného při štěpení uranu v~aktivní zóně dostatečné přirozené proudění. \par
Na reaktoru VR-1 je používáno palivo typu IRT-4M. Jedná se o~nízko-obohacené palivo, jehož dodavatelem je ruská společnost TVEL. 
Pokrytí paliva je vyrobeno z~hliníkové slitiny SAV-1. Palivové články jsou tvořeny soustavou koncentrických trubek čtvercového průřezu.
Ovládací zařízení reaktoru, které představuje systém ochrany a~řízení reaktoru, zajišťuje tyto funkce:

\begin{itemize}
    \item	měření hustoty neutronového toku (výkonu) a~rychlosti jeho relativních změn při všech provozních i~mimořádných stavech reaktoru,
    \item	ruční nebo automatické řízení hustoty neutronového toku,
    \item	nepřetržitou kontrolu stavu reaktoru se signalizací nepřípustných stavů a~přiblížení k~nim,
    \item	vlastní kontrolu před spouštěním reaktoru a~při jeho provozu se signalizací poruch,
    \item	ruční i~automatické zastavení reaktoru při jeho nepřípustných stavech i~při poruchách ovládacího zařízení.
\end{itemize}

Základními komponentami ovládacího zařízení reaktoru jsou:

\begin{itemize}
    \item	absorpční tyče,
    \item	4 kanály provozního měření výkonu (PMV),
    \item	4 kanály nezávislé výkonové ochrany (NVO),
    \item	řídicí systém, resp. řídicí počítač,
    \item	bezpečnostní řetězec,
    \item	rozhraní člověk-stroj.
\end{itemize}

Řízení štěpné reakce zajišťuje 5-7 absorpčních tyčí s~kadmiovým absorbátorem. Počet tyčí v~aktivní zóně závisí na její konkrétní 
konfiguraci. Všechny tyče jsou konstrukčně shodné, liší se pouze funkcí, určovanou podle jejich pozice v~aktivní zóně a~způsobem 
zapojení do ovládacího zařízení. V~rámci ovládacího zařízení se rozlišují 3 různé typy tyčí dle funkce: bezpečnostní, experimentální 
a~regulační. \par
Experimentální vybavení reaktoru je od svého uvedení do provozu až po současnost postupně doplňováno a~inovováno. Základním vybavením 
reaktoru, tak jako v~případě ostatních výzkumných reaktorů, jsou vertikální a~horizontální kanály. Na pracovišti reaktoru jsou
k~dispozici vertikální kanály (průměr 10 mm až 56 mm), které se umisťují buď přímo do aktivní zóny, nebo na její periferii. Vertikální 
kanály slouží především k~instalaci různých detektorů, které jsou nezávislé na ovládacím zařízení reaktoru, jsou také využívány
k~ozařování zkoumaných vzorků. Reaktor může poskytnout dále dva horizontální kanály, přičemž jeden je radiální a~druhý tangenciální. 
Tangenciální kanál, oproti radiálnímu, není pevnou součástí reaktorové nádoby, ale v~případě potřeby se do ní vkládá. 
Jak tangenciální, tak i~radiální kanál slouží k~vyvedení svazku neutronů mimo aktivní zónu reaktoru. Využívány jsou zejména
k~testování různých detekčních systémů, ozařování vzorků větších rozměrů, nebo pro účely neutronové radiografie. \par
Řada experimentálních zařízení, která se nacházejí na reaktoru, není standardní součástí výzkumných reaktorů a~byla vyvinuta 
především pro pedagogické účely, nicméně lze je využít i~pro vědecko-výzkumnou činnost. Jedná se o:

\begin{itemize}
    \item	zařízení pro studium zpožděných neutronů,
    \item	zařízení pro studium vlivu bublinkového varu na reaktivitu,
    \item	zařízení pro studium dynamiky reaktoru – vertikální oscilátor,
    \item	zařízení pro studium dynamiky reaktoru – rotační oscilátor,
    \item	zařízení pro měření teplotních efektů,
    \item	potrubní pošta.
\end{itemize}

\subsubsection{Základní parametry a~schéma reaktoru}

Základní parametry reaktoru jsou shrnuty v~tab.~\ref{tab:parametry_VR1}.
V~tabulce uvedená hodnota maximálního tepelného výkonu (500 W$_t$) je řádově nižší, než bylo stanoveno v~projektu reaktoru a~uváděno
v~předchozích bezpečnostních zprávách. Je to dáno tím, že dříve byla tato hodnota pouze odhadnuta a~nebyla experimentálně ověřena. 
Toho bylo docíleno až v~nedávné době na základě výzkumu \cite{dp_soltes_2011}, který kombinoval měření využívající aktivační analýzu 
a~výpočty programem MCNP5 \cite{standardizace_mcnp5}. Z~provedených analýz vyplynulo, že výkonu 5E8 imp./s indikovanému provozním 
měřením výkonu odpovídá hodnota tepelného výkonu přibližně 500 W$_t$, namísto původní hodnoty 5 kW$_t$.   

\begin{table}[ht!] \catcode`\-=12 % aby fungovaly cline
\renewcommand{\arraystretch}{1.2}%
\caption{Přehled základních parametrů reaktoru VR-1. \label{tab:parametry_VR1}}
\begin{tabular}{|p{6cm}|p{10cm}|}
\hline
 Výkon reaktoru dle OZ: & nominální hodnota 1E8 imp./s, maximální povolená hodnota - krátkodobě (max. 72 h ročně) 5E8 imp./s \\
\hline
 Maximální tepelný výkon reaktoru: & 500 W$_t$ \\
\hline
 Hustota toku neutronů: &  max. 5$\times 10^{10}$ n/cm$^2$ s~v~centru aktivní zóny \\
\hline
 Palivo:
 \begin{itemize}
    \item typ:
    \item palivová směs:
    \item obohacení:
    \item pokrytí:
    \item geometrie:
 \end{itemize}    
     & 
 výroba a~dovoz z~Ruské Federace 
 \begin{itemize}
    \item[]  IRT-4M
    \item[]  disperze Al+UO$_2$
    \item[]  19,7 \% $^{235}$U  
    \item[]  SAV-1 (98,4 \% Al + 1,0 \% Mg) 
    \item[]  trubkové palivo čtvercového průřezu
 \end{itemize}   \\
\hline
 Reaktorové nádoby:
 \begin{itemize}
    \item	průměr:
    \item	výška:
    \item	tloušťka stěn:
    \item	tloušťka dna:
    \item	objem:
 \end{itemize}   
& 
 vyrobeny z~nerezové oceli
 \begin{itemize}
    \item[] 2 300 mm
    \item[] 4 720 mm
    \item[] 15 mm
    \item[] 20 mm
    \item[] 17 m$^3$
 \end{itemize} \\
\hline
 Stínění:
 \begin{itemize}
    \item	nad aktivní zónou:
    \item	boční:
 \end{itemize}   
&
voda + beton
\begin{itemize}
    \item[] vrstva vody (cca 3 000 mm)
    \item[] vrstva vody (cca 850 mm) a~těžkého betonu (cca 950 mm) 
\end{itemize}\\
\hline
Moderátor a~chladivo: & lehká voda \\
\hline
 Teplota v~reaktoru: &  cca 20 °C podle teploty okolí\\
\hline
 Chlazení aktivní zóny: & přirozenou konvekcí \\
\hline
 Tlak: & atmosférický \\
\hline
 Absorpční tyče: & 5 až 7 řídicích tyčí s~kadmiovým absorbátorem\\
\hline
 Provozní měření výkonu: & čtyři širokopásmové štěpné detektory\\
 \hline
Nezávislá výkonová ochrana: & čtyři bórové korónové detektory \\
\hline
Externí zdroj neutronů: & AmBe, emise neutronů $1 \times 10^7$ n/s \\
\hline
\end{tabular}
\end{table}

Uspořádání a~hlavní komponenty reaktoru jsou zřejmé z~obr.~\ref{fig:podelny_rez} a~obr.~\ref{fig:pricny_rez}. Detailní informace o~reaktoru a~jeho komponentách jsou podány
v~kapitolách 5 až 10.

\begin{figure}[hb]
\centering
\includegraphics[width=16cm]{tex/01/VR1_podelny_rez.png}
\caption{Podélný řez reaktorem VR-1. \label{fig:podelny_rez}}
\end{figure}

\begin{figure}[hbt]
\centering
\includegraphics[width=16cm]{tex/01/VR1_pricny_rez.png}
\caption{Příčný řez reaktorem VR-1. \label{fig:pricny_rez}}
\end{figure}

\subsubsection{Neutronově-fyzikální charakteristiky reaktoru}

Základní neutronově-fyzikální charakteristiky reaktoru jsou shrnuty v~tab.~\ref{tab:NF_charakteristiky}.
Absolutní hodnota hustoty toku tepelných neutronů byla stanovena experimentálně na základě měření s~aktivačními detektory ve formě 
tenké zlaté fólie. Ostatní hodnoty byly určeny výpočtovým programem MCNP5 \cite{standardizace_mcnp5}. Uvedené hodnoty jsou charakteristické pro provozní 
konfiguraci aktivní zóny (AZ) C12 \cite{nfchar_2017}. Nicméně pro jiné konfigurace AZ reaktoru VR-1 se příliš neodlišují. Podrobné 
neutronově-fyzikální a~provozní parametry jsou pro každou provozní konfiguraci AZ zpracovány v~neutronově-fyzikálních charakteristikách
 AZ \cite{nfchar_2017}.

\begin{table}[hb!] \catcode`\-=12 % aby fungovaly cline
\renewcommand{\arraystretch}{1.3}%
\caption{Přehled základních neutronově-fyzikálních charakteristik reaktoru VR-1. \label{tab:NF_charakteristiky}}
\begin{tabular}{|p{10cm}|p{6cm}|}
\hline
Hustota toku tepelných neutronů:
\begin{itemize}
    \item	maximální hodnota:
    \item	střední hodnota:
\end{itemize}
& 
\begin{itemize}
    \item[] 1,5$\times 10^{10}$ n/cm$^2$
s~\item[] 4,3$\times 10^{9}$ n/cm$^2$
s~\end{itemize} \\
\hline
Hustota toku rychlých neutronů:
\begin{itemize}
    \item	maximální hodnota:
\end{itemize}
&
\begin{itemize}
    \item[] 1,5$\times 10^{9}$ n/cm$^2$
s~\end{itemize} \\
\hline
Hustota toku tepelných neutronů na okraji AZ: & 4,7$\times 10^{10}$ n/cm$^2$ s~\\
\hline
Střední doba života okamžitých neutronů: & 4,99$ \times 10^{-5}$ s~\\
\hline
Střední doba života zpožděných neutronů: & 9,82$ \times 10^{-2}$ s~\\
\hline
Efektivní podíl zpožděných neutronů: & 0,0077 \\
\hline
\end{tabular}
\end{table}



\subsubsection{Teplotechnické charakteristiky}	
Teplota moderátoru v~reaktorové nádobě odpovídá teplotě okolního prostředí (přibližně 20\,°C). V~důsledku nízkého výkonu reaktoru 
nedochází k~významným změnám jak v~teplotě paliva, tak i~moderátoru. Při dosažení maximálního výkonu reaktoru se změna teploty 
moderátoru v~aktivní zóně pohybuje v~oblasti desetin °C. K~odvodu tepla, vznikajícího při štěpné reakci v~aktivní zóně reaktoru, je 
dostačující přirozená konvekce. Nicméně, z~důvodu zvýšení korozní ochrany palivových článků, je v~obou reaktorových nádobách zajištěna 
cirkulace moderátoru pomocí ponorných oběhových čerpadel. \par
Teplota moderátoru na vstupu do aktivní zóny může dosahovat maximálně 45 °C, teplota povrchu palivových článků nesmí překročit 98 °C. 
Var na povrchu palivových článků není povolen. Maximální projektová hustota tepelného toku na povrchu paliva je 0,96 MW/m$^2$.

\subsubsection{Provozní stavy reaktoru}
Provozní stavy reaktoru se obecně rozdělují mezi normální a~abnormální provoz. V~rámci normálního provozu reaktoru VR-1 jsou definovány 
následující provozní stavy \cite{lap_2017}:
\begin{enumerate}
    \item	Odstavený reaktor: Stav aktivní zóny reaktoru, při němž jsou všechny výkonné prvky systému řízení a~ochrany ve stavu jejich
         maximální účinnosti a~experimentálně ověřená podkritičnost reaktoru dosahuje minimálně 3 {\texorpdfstring{$\beta$\textsubscript{ef}} {\beta ef}}.
    \item	Standardní provoz: Stav reaktoru od okamžiku zahájení spouštění reaktoru (zadáním příkazu SPUST), dosažení kritického stavu,
            zvyšování nebo snižování výkonu až do odstavení reaktoru. Výkon reaktoru může dosáhnout maximálně hodnoty stanovené ochranným systémem (7,5E8 imp./s).
    \item	Změny AZ: Stav reaktoru při změnách konfigurace AZ s~vlivem na reaktivitu větším než 0,7 {\texorpdfstring{$\beta$\textsubscript{ef}} {\beta ef}}. 
            Jedná se o~změny, které byly již ověřeny v~rámci ZKE. Tyto změny lze provádět pouze za podmínek, že experimentálně ověřená 
            podkritičnost reaktoru dosahuje minimálně 7 {\texorpdfstring{$\beta$\textsubscript{ef}} {\beta ef}}.
    \item	Základní kritický experiment: Provozní režim reaktoru, jehož cílem je stanovení určujících parametrů prvního kritického stavu nové konfigurace aktivní 
            zóny (např. počty palivových článků, polohy absorpčních tyčí, provozně uvolnitelný přebytek reaktivity). Výkon reaktoru může
v~tomto režimu dosáhnout maximálně hodnoty 1E6 imp./s.
\end{enumerate}            
Abnormálním provozem reaktoru VR-1 je stav reaktoru, který se odchyluje od normálního provozu, nicméně jeho výskyt lze očekávat, 
přičemž nevede k~závažnému poškození systémů, konstrukcí a~komponent s~vlivem na jadernou bezpečnost. Na reaktoru VR-1 se mezi tyto
 stavy řadí rychlé odstavení reaktoru a~ztráta napájení ze sítě. K~rychlému odstavení reaktoru obvykle dochází z~důvodů chybné 
 manipulace v~rámci praktické přípravy studentů. Nejčastějším důvodem je dosažení ochranné úrovně rychlosti změny výkonu nebo 
 odchylky od zadaného výkonu. Po ukončení abnormálního provozu je reaktor schopen bez opravy normálního provozu.

\subsection{Způsob provozu a~využívání reaktoru}
Reaktor je standardně v~provozu 10 měsíců v~roce, přičemž zbývající dva měsíce probíhá odstávka, jejímž předmětem je kontrola, údržba
a~popřípadě inovace komponent a~zařízení reaktoru. Hlavní provozní zátěž připadá na období zimního (říjen až prosinec) a~letního 
 (březen až květen) semestru, odstávka probíhá nejčastěji v~červenci a~srpnu. Provozní využití reaktoru se ustálilo na 900 až 1 050 h 
 za rok, což odpovídá zhruba 300 až 350 směnám. Ve dvousměnném režimu provozu to představuje 150 až 175 provozních dní. \par
Provoz reaktoru probíhá obvykle ve dvousměnném režimu (dopolední a~odpolední směna) v~pracovních dnech, přičemž jedna směna trvá 3 h.
 Po ukončení každé směny je reaktor odstaven uvedením všech absorpčních tyčí do dolních koncových poloh. \par
Hlavní část provozu reaktoru je věnována pedagogickým činnostem (výuka, výcvik a~exkurze) a~představuje přibližně 70 \% provozního 
času reaktoru. Zhruba 15 \% provozního času reaktoru jsou prováděny vědecko-výzkumné aktivity a~zbylých 15 \% náleží aktivním testům 
a~kontrolám zařízení reaktoru. \par
Výzkumné práce na reaktoru jsou limitovány jeho malým výkonem, nicméně experimentální program reaktoru se daří neustále rozšiřovat. 
Hlavní část experimentů je zaměřena na využití neutronové aktivační analýzy pro různé oblasti (archeologie, historie, mineralogie, 
potravinářství, medicína, životní prostředí apod.). Dále je reaktor využíván pro testování a~vývoj detekčních systémů, studium 
dynamiky kritických a~podkritických systémů řízených vnějším neutronovým zdrojem a~ověřování výpočetních programů.
Detailní informace o~provozu reaktoru a~jeho využívání jsou podány v~kapitolách 13 a~11.

\subsection{Nové technologie zavedené v~projektu reaktoru}
V~období let 2008 až 2018 byly na reaktoru VR-1 provedeny následující změny v~technologickém zařízení reaktoru VR-1:
\begin{itemize}
    \item	2008 - inovace systému řízení demineralizační stanice,
    \item	2009 - inovace rozhraní člověk stroj (HMI – Human Machine Interface), původní dodavatel ZAT a.s. byl nahrazen novým 
                   dodavatelem dataPartner, s.r.o.,
    \item	2009 - instalace history serveru pro sběr, záznam a~archivaci provozních dat reaktoru,
    \item	2010 - výměna mostového jeřábu,
    \item	2012 - instalace zařízení pro studium teplotních efektů do reaktorové nádoby H01,
    \item	2015 - inovace potrubní pošty reaktoru,
    \item	2015 a~2016 - výměna oběhových čerpadel pro cirkulaci moderátoru v~reaktorových nádobách,
    \item	2016 - inovace řídicího počítače.
    \item 2018 - inovace pultu operátora, instalace zařízení pro vzdálenou výuku
\end{itemize}

\subsection{Srovnání s~obdobnými projekty jaderných zařízení}
Školní reaktor VR-1 je lehkovodní jaderný reaktor bazénového typu s~obohaceným uranem. I~když takto široce charakterizovaných reaktorů
 lze najít celou řadu, identická obdoba VR-1 ani jeho přímý předchůdce neexistuje. Přestože je školní reaktor VR-1 svým způsobem 
 unikátní zařízení, byly při jeho konstrukci (s~přihlédnutím k~daným možnostem) bohatě využity zkušenosti nejen z~českých, ale
i~zahraničních reaktorových pracovišť. \par
Do značné míry měl být obdobou VR-1 rekonstruovaný reaktor ŠR-0 ve Vochově (ŠKODA Plzeň, Jaderné strojírenství), rekonstrukce však 
nebyla dokončena a~reaktor ŠR-0 byl trvale vyřazen z~provozu. \par
Lehkovodních bazénových reaktorů s~malým výkonem je několik známých typů. Zmínit lze známý typ reaktoru AGN 211 nebo ještě známější 
řadu TRIGA (podstatně vyšší výkon než VR-1 včetně možnosti pulzního provozu). \par
Pokud jde o~uspořádání bazénů (dva propojené bazény), ani toto řešení není původní a~v~různé podobě se vyskytuje u~většího počtu 
reaktorů. Poměrně typické je pro francouzské bazénové reaktory. \par
Palivo typu IRT-M, které je na reaktoru VR-1 používáno, využívá více než desítka reaktorů, prakticky vždy provozovaných na výrazně 
vyšším výkonu než VR-1. Většina z~nich pracuje na území bývalého SSSR. V~ČR na palivu obdobného typu (rozdíly jsou v~obohacení 
a~technologii, nikoliv v~geometrii) pracoval již zmíněný reaktor ŠR-0. Stejné palivo jako reaktor VR-1 používá reaktor LVR-15 v~CV Řež. \par
Shodu s~obdobnými projekty jaderných zařízení lze nalézt především v~oblasti využívání reaktoru VR-1. Svědčí o~tom kontakty, které 
pracoviště reaktoru VR-1 má např. se školním reaktorem na technické univerzitě v~Budapešti, ve Vídni a~Drážďanech.

\subsection{Systém řízení provozovatele}
V~souladu s~požadavky atomového zákona č. 263/2016 Sb. (viz § 29 odst. 1, písm. a) \cite{zakon_263_2016} je na pracovišti reaktoru VR-1 zaveden
a~udržován systém řízení. Cílem systému řízení je zajištění a~neustálé zvyšování úrovně jaderné bezpečnosti, 
radiační ochrany, technické bezpečnosti, zvládání případné radiační mimořádné události a~zabezpečení jaderného zařízení, jaderného 
materiálu a~zdroje ionizujícího záření při provozu reaktoru VR-1. V~rámci systému řízení reaktoru VR-1 je aplikován odstupňovaný 
přístup odpovídající složitosti procesů a~činností nezbytných pro provoz a~využívání reaktoru. \par
Za zavedení, koordinaci, udržování a~zlepšování systému řízení a~shody systému řízení s~vyhláškou č. 408/2016 Sb. \cite{vyhlaska_408_2016} je na 
pracovišti reaktoru VR-1 odpovědný vedoucí systému řízení. Vedoucí systému řízení je také odpovědný za plnění požadavků na 
zajišťování kvality vybraných zařízení reaktoru VR-1 v~souladu s~vyhláškou č. 358/2016 Sb. \cite{vyhlaska_358_2016}. Pro každý proces a~činnost je 
ustanovena odpovědná osoba. Ta zajišťuje, že procesy a~činnosti v~systému řízení jsou zavedeny tak, aby umožnily dosahování jeho 
cíle a~plnění všech požadavků, které mohou sloužit k~zajišťování a~zvyšování úrovně jaderné bezpečnosti, radiační ochrany, technické 
bezpečnosti, zvládání radiační mimořádné události a~zabezpečení jaderného zařízení, jaderného materiálu a~zdroje ionizujícího záření 
při provozu reaktoru VR-1. \par
Systém řízení reaktoru VR-1 je popsán v~programu systému řízení \cite{prog_syst_rizeni_2017}, přičemž konkrétní pravidla pro provádění a~řízení procesů
a~činností, včetně zvláštních procesů a~postupů řízení neshod jsou zpracována v~řídicích postupech systému řízení \cite{rp1_2017} až \cite{rp12_2017}. 
Systému řízení se podrobněji věnuje kapitola 18 této bezpečnostní zprávy.

\subsection{Zásady a~principy bezpečného provozu reaktoru}
Vysoká úroveň bezpečného provozu školního reaktoru VR-1 je dána vlastním projektem reaktoru, pravidelnými kontrolami a~údržbou zařízení 
a~komponent reaktoru, snahou o~neustále sebezlepšování a~zodpovědným a~vysoce kvalifikovaným personálem.
Projekt reaktoru zohledňuje fakt, že se jedná o~zařízení univerzitního typu, které se nachází v~hustě osídlené oblasti. Již ve stádiu 
projektování a~konstrukce reaktoru byly kladeny vysoké nároky na kvalitu zařízení a~vykonávaných stavebních a~konstrukčních prací. 
Kvalita zařízení i~provozu reaktoru je zajišťována pravidelným ověřováním technologického stavu podle programu provozních 
kontrol \cite{ppk_2017} a~plněním programu systému řízení \cite{prog_syst_rizeni_2017}. \par
Jak projekt reaktoru, tak i~způsob a~organizace jeho provozu zajišťují splnění základních principů a~zásad bezpečného využívání 
jaderné energie (viz § 45 atomového zákona \cite{zakon_263_2016}), jimiž jsou:
\begin{enumerate}[label=(\alph*)]
    \item	umožnit v~případě potřeby okamžitě a~bezpečně odstavit jaderný reaktor a~udržovat jej v~podkritickém stavu,
    \item	zabránit nekontrolovanému rozvoji štěpné řetězové reakce,
    \item	fyzikálně znemožnit vznik kritického a~nadkritického stavu mimo vnitřní prostor jaderného reaktoru,
    \item	zajišťovat odvod tepla vytvářeného jaderným palivem a~technologickými systémy
a~\item	zajistit stínění a~zabránit úniku radioaktivní látky a~šíření ionizujícího záření do životního prostředí.
\end{enumerate}
Níže jsou charakterizovány jednotlivé oblasti v~rámci nichž je zajištěn bezpečný provoz reaktoru VR-1.

\subsubsection{Jaderná a~technická bezpečnost}
Pod pojmem jaderná bezpečnost se rozumí stav a~schopnost jaderného zařízení a~fyzických osob obsluhujících jaderné zařízení zabránit 
nekontrolovatelnému rozvoji štěpné řetězové reakce nebo úniku radioaktivních látek anebo ionizujícího záření do životního prostředí
a~omezit následky nehod \cite{zakon_263_2016}. V~případě reaktoru VR-1 lze naplnění uvedené definice jaderné bezpečnosti shrnout v~následujících bodech:
\begin{itemize}
    \item	\emph{Schopnost jaderného zařízení:} reaktor byl projektován a~konstruován tak, aby plně odpovídal požadavkům, které byly kladeny 
            tehdejší legislativou \cite{vynos_9_1985} na výzkumná jaderná zařízení. Hlavním projektovým cílem bylo zajistit schopnost zařízení za 
            každé situace bezpečně zastavit štěpnou řetězovou reakci a~udržet reaktor v~podkritickém stavu. Zvláštní zřetel byl přitom 
            věnován skutečnosti, že provozovatelem reaktoru je vysoká škola a~reaktor bude ve velké míře využíván pro praktickou výuku 
            studentů. Z~tohoto vyplynuly počty a~vzájemná nezávislost kanálů měření výkonu, počty a~funkce absorpčních tyčí reaktoru, 
            neutronový zdroj, uspořádání aktivní zóny, používané jaderné palivo, způsoby manipulace v~aktivní zóně reaktoru a~jejím 
            okolí, potřebná informovanost i~pohodlí obsluhy apod. Přestože byl reaktor projektován a~konstruován v~80. letech minulého 
            století, jsou, díky pravidelným inovacím a~modernizacím, jeho součástí nejmodernější technologie. 
    \item	\emph{Stav jaderného zařízení:} ani sebelépe navržené a~vyrobené zařízení nemůže spolehlivě a~prokazatelně plnit své funkce, pokud 
            nebude dlouhodobě udržováno v~požadovaném stavu. Díky pečlivě propracované a~striktně dodržované organizaci provozu 
            reaktoru je celé zařízení, zvláště součásti a~komponenty důležité z~hlediska jaderné bezpečnosti, trvale udržováno ve 
            velmi dobrém stavu. Vyplývá to jak z~jeho náležité údržby, tak i~pravidelných provozních kontrol. Mezi ně patří zejména 
            kontroly stavu reaktorových nádob a~vnitřních částí reaktoru, stavu a~funkceschopnosti absorpčních tyčí reaktoru
a~pravidelných kontrol a~testů (při každém spouštění a~směně reaktoru a~v~rámci půlročních kontrol) systému ochran
a~řízení a~zálohové napájení.
    \item	\emph{Schopnost osob obsluhujících jaderné zařízení:} schopnostem osob obsluhujících reaktor je věnována značná pozornost. 
            Schopnost obsluhy reaktoru vychází již z~výběru vhodných pracovníků, který je založen nejen na předem daných 
            kvalifikačních kritériích, ale také jejich osobnostních charakteristikách. Schopnost obsluhy je dále zajištěna kvalitní 
            odbornou přípravou, která je každému pracovníku nastavena individuálně. Následně je schopnost obsluhy ověřována jak 
            vlastními hodnotícími systémy pracoviště, tak i~zkušební komisí jmenovanou SÚJB. Schopnost obsluhy je udržována
a~ověřována v~rámci další odborné přípravy a~periodického školení.
    \item	\emph{Stav osob obsluhujících jaderné zařízení:} osoby obsluhující reaktor mají nezastupitelnou roli při zajištění jaderné 
            bezpečnosti. Z~tohoto důvodu se obsluha reaktoru podrobuje pravidelnému ověřování své osobnostní a~zdravotní způsobilosti. 
            Zároveň jsou vytvářeny veškeré předpoklady a~podmínky, aby jejich stav odpovídal vždy vysokým nárokům, jaké klade provoz 
            jaderného zařízení na jeho obsluhu.
\end{itemize}
Zajištění technické bezpečnosti a~kvality je na pracovišti reaktoru nastaveno v~souladu s~požadavky vyhlášky č. 358/2016 Sb. \cite{vyhlaska_358_2016}. 
Zároveň jsou vypracovány postupy \cite{rp3_2017} pro zajištění technické bezpečnosti a~kvality ve všech klíčových procesech od navrhování až po 
uvádění do provozu a~provozu zařízení.  Tyto postupy se věnují jak zajištění technické bezpečnosti a~kvality vybraných, tak
i~ostatních zařízení.

\subsubsection{Radiační ochrana}
Radiační ochrana představuje systém technických a~organizačních opatření k~omezení ozáření fyzické osoby a~k~ochraně životního 
prostředí před účinky ionizujícího záření \cite{zakon_263_2016}. Podle české legislativy patří reaktor mezi velmi významné zdroje ionizujícího záření, 
je zařazen do IV. kategorie, a~tomu odpovídají i~technická a~organizační opatření, která je nutné na tomto pracovišti dodržovat. \par 
Základem technických opatření je vymezené kontrolované pásmo, stínění jak reaktoru, tak všech ostatních zdrojů ionizujícího záření, se 
kterými se na pracovišti manipuluje, radiační monitorovací systém, manipulátory a~ochranné pomůcky pro práci se zdroji ionizujícího 
záření. Stínění reaktoru je tvořeno v~radiálním směru barytovým betonem a~vrstvou vody a~v~axiálním směru vrstvou vody (cca 3 m). 
Pro práci se zdroji ionizujícího záření lze využít několik stínicích transportních kontejnerů, zároveň lze zbudovat stínění dle potřeb 
pomocí polyetylenových tvarovek s~příměsí bóru, olověných cihel nebo betonových bloků, kterých je na pracovišti reaktoru dostatečné 
množství. Radiační monitorovací systém zahrnuje monitorování pracoviště a~jeho okolí, monitorování výpustí a~systém osobního 
monitorování. Monitorování výpustí představuje nakládání s~pevnými, kapalnými a~plynnými odpady, které mohou na pracovišti reaktoru 
vzniknout a~ve kterých by mohly být obsaženy radioaktivní látky. Vzhledem k~charakteru provozu reaktoru a~dosahovanému výkonu je 
množství radioaktivních odpadů prakticky nulové. Provoz reaktoru na okolní prostředí nemá žádný vliv. \par
Mezi organizační opatření spadá ustavení dohlížející osoby, vedoucího radiační ochrany a~osob s~přímým dohledem nad radiační 
ochranou. Dále je to odborná příprava vybraných pracovníků, pravidelná školení radiačních pracovníků a~ověřování jejich odborné 
způsobilosti. Důležitou součástí organizačních opatření je systém nakládání se zdroji ionizujícího záření tak, aby zdroje byly 
používány přiměřeně a~nakládaly s~nimi pouze osoby k~tomu určené. S~ohledem na způsob využívání reaktoru je klíčová pečlivá příprava 
experimentálních činností na tomto zařízení. Zde je nezbytná optimalizace radiační ochrany v~případě každého nově zaváděného 
experimentu, resp. činnosti, které by mohly radiační ochranu ovlivnit. \par
Bližší informace o~zajištění radiační ochrany na pracovišti reaktoru VR-1 jsou uvedeny v~kapitole 12.

\subsubsection{Zabezpečení jaderného zařízení, jaderného materiálu a~radionuklidového zdroje}

Zabezpečení jaderného zařízení, jaderného materiálu a~radionuklidového zdroje na pracovišti reaktoru vychází z~definice fyzické 
ochrany, která představuje systém technických a~organizačních opatření zabraňující neoprávněným činnostem s~jaderným zařízením nebo 
jaderným materiálem \cite{zakon_263_2016}. \par
Z~hlediska technického zabezpečení lze pracoviště rozdělit na plášťovou a~prostorovou ochranu. Pláštěm chráněného prostoru se rozumí 
stavební vymezení oblasti haly reaktoru. Plášť je ze severní strany tvořen prosklenou stěnou s~vjezdovými vraty a~dále obvodovými 
zdmi haly, stropem a~podlahou. Hlavní vchod a~nouzové východy, resp. vstup do skladu paliva jsou opatřeny uzamykatelnými bezpečnostními
 dveřmi. Elektronická zabezpečovací signalizace (EZS) se skládá zejména z~ústředny a~prvky detekce narušení pláště a~detekce pohybu 
 uvnitř chráněného prostoru. EZS je spojena pomocí vysílače a~telefonní linky se Systémem centralizované ochrany Policie ČR a~dále 
 je vyveden výstup z~ústředny EZS do GSM brány informující pracovníky v~pracovní pohotovosti o~případném narušení. Součástí systému
  zabezpečení reaktoru je i~televizní okruh, jehož kamery snímají prostor haly a~její bezprostřední okolí. Kamery jsou propojeny se 
  záznamovým zařízením a~monitorem ve velínu reaktoru. \par
Z~pohledu organizačního je většina pracovníků reaktoru s~oprávněním samostatného vstupu do chráněného prostoru zároveň pracovníky 
vykonávající citlivé činnosti ve smyslu § 162, odst. 2, Atomového zákona a~jsou držiteli platného dokladu o~bezpečnostní způsobilosti. 
Ostatní pracovníci musí splňovat minimálně podmínky pro přístup k~utajovaným informacím na stupni Vyhrazené. Všechny další vstupující
 osoby přicházející na pracoviště reaktoru jsou v~nezbytné míře o~fyzické ochraně pracoviště prokazatelně poučeny a~jsou doprovázeny 
 pracovníkem s~oprávněním samostatného vstupu. Vstup osob na pracoviště reaktoru je evidován, povolení vstupu smějí vydat pouze 
 oprávněné osoby a~před vstupem na pracoviště je prováděna kontrola dodržování zákazu vstupu se zbraněmi, výbušninami a~mobilními 
 telefony. Pracovníci reaktoru jsou pravidelně jednou ročně a~při nástupu do pracovního poměru prokazatelně poučeni o~svých 
 povinnostech vyplývajících ze zabezpečení pracoviště.

\subsubsection{Připravenost k~odezvě na radiační mimořádnou událost}
Připravenost k~odezvě na radiační mimořádnou událost představuje soubor organizačních, technických, materiálních a~personálních 
opatření připravovaných podle pravděpodobného průběhu radiační mimořádné události k~odvrácení nebo zmírnění jejích dopadů
zpracovaných ve formě zásahových instrukcí, vnitřního havarijního plánu, havarijního řádu, plánu k~provádění záchranných
a~likvidačních prací v~okolí zdroje nebezpečí a~národního radiačního havarijního plánu \cite{zakon_263_2016}. \par
Způsob zajištění (technicky i~organizačně) zvládání radiačních mimořádných událostí reaktoru je podrobně popsán ve vnitřním 
havarijním plánu reaktoru. Tento plán vychází z~dlouholetých zkušeností s~havarijní připraveností na pracovišti školního reaktoru 
VR-1. Vnitřní havarijní plán je pravidelně aktualizován a~jsou v~něm zohledněny jak požadavky platné legislativy, tak
i~zkušenosti získávané využíváním a~provozem reaktoru. \par
Z~analýz provedených na pracovišti reaktoru plyne, že při provozu reaktoru mohou nastat radiační mimořádné události 1. stupně
a~radiační nehody. Radiační havárie je na pracovišti reaktoru vyloučena. Z~celkového počtu šesti uvažovaných radiačních mimořádných 
událostí popsaných ve vnitřním havarijním plánu jsou čtyři radiační mimořádné události 1. stupně  a~dvě radiační nehody. \par 
Mezi radiační mimořádné události prvního stupně se řadí:
\begin{itemize}
    \item	selhání řídicího systému reaktoru doprovázené nekontrolovaným zvýšením výkonu,
    \item	poškození palivových článků nebo neutronového zdroje,
    \item	únik vody z~vodního hospodářství reaktoru,
    \item	ztráta kontroly nad zdrojem.
\end{itemize}
Mezi radiační nehody patří:
\begin{itemize}
    \item	požár reaktorového pracoviště,
    \item	nehoda způsobená vnějšími vlivy.
\end{itemize}   
Organizačně je odpovědný za připravenost k~odezvě na radiační mimořádnou událost vedoucí havarijní připravenosti, který zajišťuje 
plnění požadavků vyhlášky č. 359/2016 Sb. \cite{vyhlaska_359_2016}. 
Podrobnější informace o~připravenosti pracoviště k~odezvě na radiační mimořádnou událost poskytuje kapitola 20.

\clearpage





